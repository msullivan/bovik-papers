\documentclass{sigplanconf}
\usepackage[colorlinks=true,linkcolor=black,urlcolor=black,citecolor=black]%
{hyperref}
\usepackage{graphicx}
\usepackage{verbatim}
\usepackage{listings}
\usepackage[usenames,dvipsnames]{color}

\copyrightyear{2012}
\copyrightdata{}

\newcommand{\prettybox}[1]{\fbox{\parbox{\linewidth}{#1}}}
\newenvironment{smallverbatim}{\endgraf\small\verbatim}{\endverbatim} 
\renewcommand{\t}{\texttt}
\renewcommand{\b}{\textbf}
\renewcommand{\i}{\textit}

\lstset{
  language=Haskell,
  basicstyle=\small\ttfamily,
  frame=trbl,
  basewidth={0.5em,0.45em},
  commentstyle=\ttfamily\color{ForestGreen},
  literate=*{+}{{$+$}}1 {/}{{$/$}}1 {*}{{$*$}}1 {=}{{$=$}}1
            {>}{{$>$}}1 {<}{{$<$}}1 {\\}{{$\lambda$}}1
            {\\\\}{{\char`\\\char`\\}}1
            {->}{{$\rightarrow$}}2 {>=}{{$\geq$}}2 {<-}{{$\leftarrow$}}2
            {<=}{{$\leq$}}2 {=>}{{$\Rightarrow$}}2
            %{\ .}{{$\circ$}}2 {\ .\ }{{$\circ$}}2
            {>>}{{>>}}2 {>>=}{{>>=}}2
            {|}{{$\mid$}}1
}
%\lstnewenvironment{code}{\lstset{frame=}}{}
\newenvironment{code}{\endgraf\small\verbatim}{\endverbatim} 

\title{The Spineless Tagless Tweet Machine}
\subtitle{Distributed Cloud-Based Social Crowdsourced Lazy Graph Reduction on the Web 2.0}
\authorinfo{Michael Sullivan}{Carnegie Mellon University}{mjsulliv@cs.cmu.edu}

\begin{document}
\maketitle

\begin{abstract}
Lazy graph reduction is a common technique for implementing non-strict
functional programming languages.
We present the Spineless Tagless Tweet Machine, a distributed lazy
graph reduction system that uses Twitter to communicate and store
unevaluated expressions.
\end{abstract}

\section{Introduction}
Over the last few decades, there has been a large amount of work
discussing methods for efficient implementation of non-strict
functional programming languages \cite{Jones92implementinglazy}
\cite{Naylor:2010:RR:1932681.1863556}. 
%One traditional method of implementing non-strict evaluation is lazy
%graph reduction.
A much touted benefit of non-strict, pure functional languages is that
computations can be easily parallelized without worrying about
concurrency issues.

We propose that the lazy graph reduction model combines well with a
number of other major recent developments in the computing world: the
emergence of cloud computing and the social web. Cloud computing
refers to the offloading of computation and storage to unaccountable
and untrusted software-as-a-service providers, while the social web
refers to the ``sharing'' of statuses, photographs, personal
information, and other data with friends, acquaintances, and enemies
(frequently in the form of ``frenemies'') through the Web.  One of the
most popular cloud applications on the social web is Twitter
\cite{twitter}, which allows users to publish 140-character ``tweets''
(which include images), search for tweets by ``hashtags'', and other
such ``social'' activities.

Our proposal is to take advantage of of the parallelizability of lazy
pure functional programs by using the tools of cloud computation and
the social web.  We present the Spineless Tagless Tweet Machine, a
distributed lazy graph reduction system that uses Twitter to share
unevaluated expressions with the user's friends, acquaintances, and
frenemies.

\section{Description}

Lazy computation is generally implemented by creating ``thunks''
containing computations that might be needed later and evaluating the
thunks (or ``pulling on'' them) only when the result is
needed. Pulling on a thunk may involve creating new thunks
representing computations for subparts of the value computed.

The Spineless Tagless Tweet Machine evaluates a internal language
called \t{STTM}. The \t{STTM} is an austere untyped internal language
with support for lazy evaluation. \t{STTM} is essential a more
syntactically restricted variant of the untyped lambda calculus.
Algebraic data-types are represented
using the Scott encoding; that is, as functions that take case
functions for each of their branches and then call the matching one.
Integers and integer operations are included in the language for
efficiency.

In the Spineless Tagless Tweet Machine model, interested users run
Spineless Tagless Tweet Machine computation nodes on their
machines. These nodes monitor twitter for \t{STTM} tweets containing
unevaluated thunks and upon seeing them may choose to do some
evaluation of them and tweet the results. This may involve creating
additional thunks. Note that this does not necessarily need to been
done by the \t{STTM} software: individual users should feel encouraged
to evaluate thunks by hand and tweet the results, allowing a
crowd-sourced graph reduction.

Each unevaluated expression will be associated with a unique
hash-tag \footnote{For this reason, ``tagless'' is
  inaccurate. ``Spineless'' also is inaccurate.}  To pull on thunks, a
client searches for tweets with the appropriate hashtag, in order to
find an evaluated version. (This search may need to be repeated until
the thunk has been evaluated.) Since the size of expressions is likely
to exceed the strict 140-character limit, expressions will be encoded
as images and included with the tweets.

\section{Characteristics}
One major advantage of this system is that caching of results is
provided by Twitter. If a computation has been performed before, the
result will be immediately found when pulling on the thunk.

Some potential downsides in this system are the potential weakness in
privacy and correctness. The system provides no mechanism to assess
the correctness of evaluated tweets or to hide tweets containing
private computational data from unwanted observers. In practice, users
of cloud and social services seem to not mind these limitations.

Deciding which tweets should be evaluated remains a major open problem
in this work. Only evaluating tweets that are directly needed fails to
gain any parallelism, while evaluating all tweets will result in
chasing down infinite data structures and rapidly being rate-limited
by Twitter.


\section{Conclusion}
In this paper, we presented a novel way to violate Twitter's Terms of
Service \cite{twitterTOS} by using it as the communication medium and
backing store for distributed lazy graph reduction.
We would have evaluated the performance, but will be implementing the
system sometime between now and the conference.


\bibliography{citations}{}
\bibliographystyle{abbrvnat}

\end{document}
