\documentclass[preprint]{sigplanconf}
\usepackage[colorlinks=true,linkcolor=black,urlcolor=black,citecolor=black]%
{hyperref}
\usepackage{graphicx}
\usepackage{verbatim}
\usepackage{listings}
\usepackage[usenames,dvipsnames]{color}

\newcommand{\prettybox}[1]{\fbox{\parbox{\linewidth}{#1}}}
\newenvironment{smallverbatim}{\endgraf\small\verbatim}{\endverbatim} 
\renewcommand{\t}{\texttt}
\renewcommand{\b}{\textbf}
\renewcommand{\i}{\textit}

\lstset{
  language=Haskell,
  basicstyle=\small\ttfamily,
  frame=trbl,
  basewidth={0.5em,0.45em},
  commentstyle=\ttfamily\color{ForestGreen},
  literate=*{+}{{$+$}}1 {/}{{$/$}}1 {*}{{$*$}}1 {=}{{$=$}}1
            {>}{{$>$}}1 {<}{{$<$}}1 {\\}{{$\lambda$}}1
            {\\\\}{{\char`\\\char`\\}}1
            {->}{{$\rightarrow$}}2 {>=}{{$\geq$}}2 {<-}{{$\leftarrow$}}2
            {<=}{{$\leq$}}2 {=>}{{$\Rightarrow$}}2
            %{\ .}{{$\circ$}}2 {\ .\ }{{$\circ$}}2
            {>>}{{>>}}2 {>>=}{{>>=}}2
            {|}{{$\mid$}}1
}
%\lstnewenvironment{code}{\lstset{frame=}}{}
\newenvironment{code}{\endgraf\small\verbatim}{\endverbatim} 

\title{The Spineless Tagless Tweet Machine}
\subtitle{Distributed Cloud-Based Social Crowdsourced Lazy Graph Reduction on the Web 2.0}
\authorinfo{Michael Sullivan}{Carnegie Mellon University}{mjsulliv@cs.cmu.edu}

\begin{document}
\maketitle

\begin{abstract}

Over the last few decades, there has been a large amount of work
discussing methods for efficient implementation of non-strict
functional programming languages \cite{Jones92implementinglazy}
\cite{Naylor:2010:RR:1932681.1863556}. One traditional method of
implementing non-strict evaluation is lazy graph reduction. A much
touted benefit of non-strict, pure functional languages is that
computations can be easily parallelized without worrying about
concurrency issues.

We propose that the lazy graph reduction model combines well with a
number of other major recent developments in the computing world: the
emergence of cloud computing and the social web. Cloud computing
refers to the offloading of computation and storage to unaccountable
and untrusted software-as-a-service providers, while the social web
refers to the ``sharing'' of statuses, photographs, personal
information, and other data with friends, acquaintances, and enemies
(frequently in the form of ``frenemies'') through the Web.



\end{abstract}

\section{Introduction}


\bibliography{citations}{}
\bibliographystyle{abbrvnat}

\end{document}
